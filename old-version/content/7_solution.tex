En este capítulo se planteara la solución propuesta, indicando los pasos a seguir, porqué se escogió esta solución y las principales dificultades que se encontraron al momento de implementarla. 

Antes de entrar a las secciones siguientes, se deberá explicar el global de la solución propuesta, indicando los resultados esperados y como estos, pueden resolver la problemática planteada.

\section{Topic Modeling}
\label{sec:soltopicmodeling}
Explicar como se ejecuto el topic modeling, explicando los 3 (o 4) pasos que se utilizaron. Además, sería interesante mostrar los tiempos de ejecución para ver las consecuencias del problema de la dimensionalidad presentado.

Además, mencionar porque se eligió realizar esos pasos y no otros, como por ejemplo Stemming de palabras. O usar un pre-procesamiento a la data directamente.

\section{FCA}
\label{sec:solfca}
En esta sección explicar cual fue el proceso para ejecutar FCA. Mencionar las herramientas utilizadas: \emph{Coron} y \emph{Sephirot}. Además, mencionar que lista de tokens fue utilizada junto con indicar el método que se utilizo para seleccionar el parámetro de minimal support.

\section{Visualización}
\label{sec:solvisualization}
Explicar porque se eligió trabajar con Sigma.JS. Las dificultades presentadas, cuales son los beneficios de esta herramienta frente a otros.