En este capítulo se describirá los problemas que se buscan resolver en profundidad, resaltando la importancia de encontrar una solución y las consecuencias que tendrían estas soluciones.

\section{Base de Datos Científicas}
\label{sec:sciencedb}
Explicar porque las bases de datos científicas son tan importantes. Como se clasifican las que mayor resaltan: DBLP y JSTOR. Porque es importante tener un buscador que permita \emph{navegar} en la colección de conocimiento y como este mismo se descubre a partir de una búsqueda bien realizada.

\section{Dimensionalidad}
\label{sec:dimensionality}
Muchas de las herramientas mencionadas: \emph{JSTOR} y \emph{DBLP} tienen grandes colecciones de datos. Esto provoca que el procesamiento de estas bibliotecas científicas tengan una dificultad intrínseca: la alta dimensionalidad de la información.

\section{Visualización de Lattices}
\label{sec:latticevisualization}
Cómo se planteó en el problema de las base de datos científicas. No solo basta con procesar y clasificar estas grandes colecciones de documentos, sino que también, es necesario desarrollar una aplicación para que el usuario pueda navegar por esta clasificación encontrando y descubriendo los resultados que le parezcan interesantes de acuerdo a su búsqueda