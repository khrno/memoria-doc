% \documentclass[letterpaper,12pt]{article}
\documentclass[letterpaper,12pt]{book}
\usepackage{amsmath}
\usepackage{amssymb}
\usepackage{amsfonts}
\usepackage{array}
\usepackage{longtable}

\usepackage{framed}
% \usepackage{subfig}
\usepackage{multirow}
\usepackage{hyperref}

%% graphics and images
\usepackage{lscape}
\usepackage{graphicx}
\usepackage{graphics}
\usepackage{rotating}
\usepackage{caption}
\usepackage{subcaption}
% \usepackage{multirow}

\usepackage{setspace}
\usepackage{titlesec}

%% Algorithm package
% \usepackage{algorithm}
\usepackage[chapter]{algorithm}
\usepackage[noend]{algpseudocode}

%% Algorithm custom command
\makeatletter
\def\BState{\State\hskip-\ALG@thistlm}
\makeatother

%% verbatim samepage
\usepackage{fancyvrb}

\usepackage{xspace}

%% \usepackage{natbib}
%% \bibliographystyle{abbrvnat}
%% \bibpunct{(}{)}{;}{a}{,}{,}
\usepackage[utf8]{inputenc}

\usepackage[spanish]{babel}

\setlength\topmargin{-0.375in}
\setlength\headheight{0in}
\setlength\headsep{0.375in}
\setlength\textheight{9.5in}
\setlength\textwidth{7.0in}
\setlength\oddsidemargin{-0.25in}
\setlength\evensidemargin{-0.25in}
\setlength\parindent{0.25in}
\setlength\parskip{0.25in}

%% \usepackage{amssymb}
%% using square for first level listing instead circles
\renewcommand{\labelitemi}{\tiny$\blacksquare$}

%% \usepackage{cite}
\usepackage[backend=bibtex, bibencoding=utf8]{biblatex}

\usepackage[letterpaper, portrait, margin=1.25in]{geometry}


\providecommand{\lessgtr}{\stackrel{<}{>}} %% could be provided natively with different font encoding:
\IfFileExists{pslatex.sty}{

  \usepackage{pslatex}
}{}

%% Custom commands
\newcommand{\eng}[1]{\textit{#1}\xspace}


\newcolumntype{M}[1]{>{\centering\arraybackslash}m{#1}}
\newcolumntype{N}{@{}m{0pt}@{}}




\DefineBibliographyStrings{english}{
  references = {Bibliografía},
}
\bibliography{references.bib}


\titleformat{\chapter}[display]
{\normalfont\huge\bfseries}{Capítulo \ \thechapter}{5pt}{\Huge}   
\titlespacing*{\chapter}{0pt}{-50pt}{40pt}
\titlespacing*{\section}{0pt}{0pt}{0pt}



\begin{document}

\begin{titlepage}
\newgeometry{top=3cm}
\begin{centering}
{
\textbf{\Large UNIVERSIDAD TÉCNICA FEDERICO SANTA MARÍA}\\[0em]
\textbf{\small DEPARTAMENTO DE INFORMÁTICA}\\[0em]
\textbf{\small SANTIAGO -- CHILE}\\[2.4em]
}
\includegraphics[scale=0.295]{content/figures/logo-utfsm} \\[4em]
\begin{spacing}{1.8}
\textbf{\Large \uppercase{``Título de la memoria''}}\\[3.70em]
\end{spacing}
\textbf{\large \uppercase{Pablo Antonio Ortega Mesa}}\\[0.54em]
\textbf{\small MEMORIA DE TITULACIÓN PARA OPTAR AL TÍTULO DE INGENIERO CIVIL INFORMÁTICA} \\[1em]
\begin{tabular}{ccc} \small\bf
\small\bf PROFESOR GUÍA:          & & \small\bf \uppercase{Marcelo Mendoza} \\
\small\bf PROFESOR CORREFERENTE:  & & \small\bf \uppercase{Profesor Correferente} \\
\end{tabular} \\[2.5em]
\textbf{\small Mayo -- 2015} \\
\end{centering}
\end{titlepage}


\setstretch{1.5}

\section*{Resumen}
\input{content/1_abstract_es.tex}
\clearpage

\section*{Abstract}
\input{content/2_abstract_en.tex}
\clearpage

\section*{Agradecimientos}
\input{content/3_acknowledgments.tex}
\clearpage

\let\cleardoublepage\clearpage

%% comment it out if you don't like it and/or use one of the lines below:
%% \setstretch{2}
%\setstretch{1.5}

\tableofcontents
\clearpage

% \listoffigures
% \clearpage

% \listoftables
% \clearpage

% \newpage
\chapter{Introducción}
\label{chap:intro}
En esta sección se propone escribir sobre la importancia del descubrimiento de conocimiento por parte de la comunidad científica. El principal problema para esta tarea es la búsqueda dentro de millones de documentos científicos sin una clasificación correcta con el objetivo de obtener la literatura relacionada con las investigaciones en curso. 

Por otro lado se planteará la importancia de disponer de una correcta visualización para la clasificación de documentos y las utilidades de esta.

Se hará referencia las técnicas FCA y LDA. Además se mencionará el uso que se le da en este trabajo.

Finalmente, se planteará la propuesta de este trabajo y se plantearán los objetivos.

\clearpage

\chapter{Marco Teórico}
\label{chap:mteorico}
En este capítulo se hablará sobre el marco teórico de las técnicas utilizadas: FCA y LDA. Además de hacer un breve análisis de los trabajos relacionados a cada una de estas.

\section{Formal Concept Analysis}
\label{sec:fca}

En esta sección se planteará todo el framework matemático que constituye FCA. Luego, se realizará un breve análisi de la literatura asociada. (NOTA: utilizar la referencia utilizada por Víctor Codocedo en presentación, ya que incluye los principales ``milestones'' de la técnica).

\section{Topic Modeling}
\label{sec:topicmodeling}

En esta sección se describirá en que consiste Topic Modeling, sus objetivos y como nace. Haciendo una breve pasada por \emph{PSLI}.

\subsection{Modelo LDA}
\label{subsec:ldamodel}
Se explicará en detalle en que consiste el modelo LDA, presentando ejemplos para la mejor comprensión del lector.

\subsection{Inferencia Estadística}
\label{subsec:infest}
Para continuar explicando \emph{Modelo LDA}

\subsection{Gibbs Sampling}
\label{subsec:gibbssampling}
Se explicará la técnica \emph{Gibbs Sampling} ya que es la utilizada por los algoritmos de este trabajo.

\section{DBLP}
\label{sec:dblp}
DBLP es una gran colección de trabajos científicos muy importante para la comunidad. Se explicará la importancia, los desafíos que arroja al ser una ``\emph{Gran Colección}'' y el porqué se escogió esta base de datos.
\clearpage

\chapter{Descripción del Problema}
\label{chap:problem}
En este capítulo se describirá los problemas que se buscan resolver en profundidad, resaltando la importancia de encontrar una solución y las consecuencias que tendrían estas soluciones.

\section{Base de Datos Científicas}
\label{sec:sciencedb}
Explicar porque las bases de datos científicas son tan importantes. Como se clasifican las que mayor resaltan: DBLP y JSTOR. Porque es importante tener un buscador que permita \emph{navegar} en la colección de conocimiento y como este mismo se descubre a partir de una búsqueda bien realizada.

\section{Dimensionalidad}
\label{sec:dimensionality}
Muchas de las herramientas mencionadas: \emph{JSTOR} y \emph{DBLP} tienen grandes colecciones de datos. Esto provoca que el procesamiento de estas bibliotecas científicas tengan una dificultad intrínseca: la alta dimensionalidad de la información.

\section{Visualización de Lattices}
\label{sec:latticevisualization}
Cómo se planteó en el problema de las base de datos científicas. No solo basta con procesar y clasificar estas grandes colecciones de documentos, sino que también, es necesario desarrollar una aplicación para que el usuario pueda navegar por esta clasificación encontrando y descubriendo los resultados que le parezcan interesantes de acuerdo a su búsqueda
\clearpage

\chapter{Solución Propuesta}
\label{chap:solution}
En este capítulo se planteara la solución propuesta, indicando los pasos a seguir, porqué se escogió esta solución y las principales dificultades que se encontraron al momento de implementarla. 

Antes de entrar a las secciones siguientes, se deberá explicar el global de la solución propuesta, indicando los resultados esperados y como estos, pueden resolver la problemática planteada.

\section{Topic Modeling}
\label{sec:soltopicmodeling}
Explicar como se ejecuto el topic modeling, explicando los 3 (o 4) pasos que se utilizaron. Además, sería interesante mostrar los tiempos de ejecución para ver las consecuencias del problema de la dimensionalidad presentado.

Además, mencionar porque se eligió realizar esos pasos y no otros, como por ejemplo Stemming de palabras. O usar un pre-procesamiento a la data directamente.

\section{FCA}
\label{sec:solfca}
En esta sección explicar cual fue el proceso para ejecutar FCA. Mencionar las herramientas utilizadas: \emph{Coron} y \emph{Sephirot}. Además, mencionar que lista de tokens fue utilizada junto con indicar el método que se utilizo para seleccionar el parámetro de minimal support.

\section{Visualización}
\label{sec:solvisualization}
Explicar porque se eligió trabajar con Sigma.JS. Las dificultades presentadas, cuales son los beneficios de esta herramienta frente a otros.
\clearpage

\chapter{Resultados}
\label{chap:results}
En este capítulo se presentarán, señalarán, explicarán y analizarán los resultados obtenidos en el trabajo.

Por ejemplo, la calidad de los tópicos descubiertos a través de LDA son los siguientes:

\begin{enumerate}
\item Tópico 1
\item Tópico 2
\item ...
\end{enumerate}


Lorem ipsum dolor sit amet, consectetur adipisicing elit, sed do eiusmod
tempor incididunt ut labore et dolore magna aliqua. Ut enim ad minim veniam,
quis nostrud exercitation ullamco laboris nisi ut aliquip ex ea commodo
consequat. Duis aute irure dolor in reprehenderit in voluptate velit esse
cillum dolore eu fugiat nulla pariatur. Excepteur sint occaecat cupidatat non
proident, sunt in culpa qui officia deserunt mollit anim id est laborum.


Lorem ipsum dolor sit amet, consectetur adipisicing elit, sed do eiusmod
tempor incididunt ut labore et dolore magna aliqua. Ut enim ad minim veniam,
quis nostrud exercitation ullamco laboris nisi ut aliquip ex ea commodo
consequat. Duis aute irure dolor in reprehenderit in voluptate velit esse
cillum dolore eu fugiat nulla pariatur. Excepteur sint occaecat cupidatat non
proident, sunt in culpa qui officia deserunt mollit anim id est laborum.


Lorem ipsum dolor sit amet, consectetur adipisicing elit, sed do eiusmod
tempor incididunt ut labore et dolore magna aliqua. Ut enim ad minim veniam,
quis nostrud exercitation ullamco laboris nisi ut aliquip ex ea commodo
consequat. Duis aute irure dolor in reprehenderit in voluptate velit esse
cillum dolore eu fugiat nulla pariatur. Excepteur sint occaecat cupidatat non
proident, sunt in culpa qui officia deserunt mollit anim id est laborum.



\clearpage

\chapter{Trabajo futuro}
\label{chap:futurework}
\section*{Trabajo futuro 1}

Análisis de la red social generada por la biblioteca de documentos científicos. Existen dos redes sociales:

\begin{enumerate}
\item Red social formada por los autores y co-autores de las publicaciones.
\item Red social formada por las citaciones de las publicaciones.
\end{enumerate}


\section*{Trabajo futuro 2}

Lorem ipsum dolor sit amet, consectetur adipisicing elit, sed do eiusmod
tempor incididunt ut labore et dolore magna aliqua. Ut enim ad minim veniam,
quis nostrud exercitation ullamco laboris nisi ut aliquip ex ea commodo
consequat. Duis aute irure dolor in reprehenderit in voluptate velit esse
cillum dolore eu fugiat nulla pariatur. Excepteur sint occaecat cupidatat non
proident, sunt in culpa qui officia deserunt mollit anim id est laborum.
\clearpage

\chapter{Conclusiones}
\label{chap:concl}
\input{content/10_conclusions.tex}
\newpage

% \bibliographystyle{plain}
\setstretch{1}
\nocite{*}
\printbibliography

\end{document}