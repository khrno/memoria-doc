En este capítulo se hablará sobre el marco teórico de las técnicas utilizadas: FCA y LDA. Además de hacer un breve análisis de los trabajos relacionados a cada una de estas.

\section{Formal Concept Analysis}
\label{sec:fca}

En esta sección se planteará todo el framework matemático que constituye FCA. Luego, se realizará un breve análisi de la literatura asociada. (NOTA: utilizar la referencia utilizada por Víctor Codocedo en presentación, ya que incluye los principales ``milestones'' de la técnica).

\section{Topic Modeling}
\label{sec:topicmodeling}

En esta sección se describirá en que consiste Topic Modeling, sus objetivos y como nace. Haciendo una breve pasada por \emph{PSLI}.

\subsection{Modelo LDA}
\label{subsec:ldamodel}
Se explicará en detalle en que consiste el modelo LDA, presentando ejemplos para la mejor comprensión del lector.

\subsection{Inferencia Estadística}
\label{subsec:infest}
Para continuar explicando \emph{Modelo LDA}

\subsection{Gibbs Sampling}
\label{subsec:gibbssampling}
Se explicará la técnica \emph{Gibbs Sampling} ya que es la utilizada por los algoritmos de este trabajo.

\section{DBLP}
\label{sec:dblp}
DBLP es una gran colección de trabajos científicos muy importante para la comunidad. Se explicará la importancia, los desafíos que arroja al ser una ``\emph{Gran Colección}'' y el porqué se escogió esta base de datos.